\documentclass[a4paper,10pt]{article}
\usepackage[paper=a4paper, hmargin=1.5cm, bottom=1.5cm, top=3.5cm]{geometry}
\usepackage[spanish]{babel}
\usepackage[utf8]{inputenc}
%\usepackage{fancyhdr}
%\usepackage{charter} % tipografia
%\usepackage{graphicx}
%\usepackage{makeidx}
%\usepackage{float}
%\usepackage{amsmath, amsthm, amssymb}
%\usepackage{amsfonts}
%\usepackage{sectsty}
%\usepackage{charter}
%\usepackage{wrapfig}
%\usepackage{listings}
%\lstset{language=C}
%\input{codesnippet}
%\input{page.layout}
% \setcounter{secnumdepth}{2}
%\usepackage{underscore}
\usepackage{caratula}
%\usepackage{url}


\begin{document}
\thispagestyle{empty}
\materia{Organizaci\'on del Computador II}
\submateria{Primer Cuatrimestre de 2015}
\titulo{Trabajo Práctico Filtros}
\subtitulo{Grupo: Super\_Smash\_Bros\_/\_Nintendo\_64}
\integrante{Arribas, Joaqu\'in}{702/13}{joacoarribas@hotmail.com}
\integrante{V\'azquez, J\'esica}{318/13}{jesis\_93@hotmail.com}
\integrante{Zavalla, Agust\'in}{670/13}{nkm747@gmail.com}

\maketitle
\newpage

\section{Objetivos Generales}
El trabajo práctico consiste en una serie de ejercicios en los que se aplican de forma gradual los conceptos necesarios para construir un sistema operativo mínimo.
El sistema será capaz de capturar cualquier excepción que puedan generar las tareas y tomar las acciones necesarias para quitar a esta del sistema.

\section{Introducción}
El trabajo práctico consiste en la implementación de un juego de dos jugadores llamado \textbf{Tierra pirata}.
Los piratas son considerados tareas del sistema operativo que se mueven por la memoria.

\section{Ejercicios}

\subsection{Ejercicio 1}
Pasos que realizamos para hacer el ejercicio 1:
\begin{enumerate}
  \item Creamos la tabla GDT (Global Descriptor Table) con 5 descriptores de segmentos de memoria, uno nulo, dos para código de nivel 0 y 3, y dos para datos de nivel 0 y 3.
  \item Direccionamos con cada uno de estos segmentos los primeros 500MB de memoria.
  \item Deshabilitamos las interrupciones.
  \item Habilitamos A20.
  \item Cargamos el registro GDTR con la direccion base de la GDT, utilizando la instrucción LGDT.
  \item Seteamos el bit PE del registro CR0 en 1.
  \item Pasamos a modo protegido haciendo un jump a la etiqueta \textbf{modoProtegido}.
  \item Seteamos los selectores de segmentos. (A completar mejor, yo no saber nada)
  \item Seteamos la base de la pila en la dirección de memoria 0x27000.
\end{enumerate}

\end{document}
